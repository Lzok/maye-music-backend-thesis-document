%!TEX root = ../thesis.tex
%*******************************************************************************
%*********************************** First Chapter *****************************
%*******************************************************************************

\chapter{Introducción}  %Title of the First Chapter

\ifpdf
    \graphicspath{{Chapter1/Figs/Raster/}{Chapter1/Figs/PDF/}{Chapter1/Figs/}}
\else
    \graphicspath{{Chapter1/Figs/Vector/}{Chapter1/Figs/}}
\fi


%********************************** %First Section  **************************************
\section{Surgimiento de la musica} %Section - 1.1 

El origen de la música es desconocido, ya que inicialmente no se utilizaban instrumentos
musicales para interpretarla, solo la voz humana, o la percusión corporal, que no dejan
huella en el registro arqueológico. 

Dado que toda cultura conocida ha tenido alguna forma de manifestación musical, la
historia de la música abarca a todas las sociedades y épocas. Se puede inferir que la
música se descubrió en un momento similar a la aparición del lenguaje. El cambio de altura
musical en el lenguaje produce un canto, de manera que es probable que en los orígenes
apareciera de esta manera.

La música de una determinada sociedad está estrechamente relacionada con ciertos aspectos
de su cultura, tales como su organización económica, su desarrollo tecnológico,
tradiciones, creencias religiosas, etcétera. En casi todas las culturas se considera a la
música como un regalo de los dioses. En la Antigua Grecia se consideraba a Hermes como el
transmisor de la música a los humanos, y primer creador de un instrumento musical, el
arpa, al tender cuerdas sobre el caparazón de una tortuga. Hace unos cinco mil años, un
emperador en China, ordenó crear la música a sus súbditos, y les dijo que para ello debían
de basarse en los sonidos de la naturaleza. Entre la mitología germánica se cree que
Heimdall, tenía un cuerno gigantesco que debía tocar cuando comenzara el crepúsculo de los
dioses. Las leyendas son similares para el resto de culturas primitivas, tanto perdidas
como modernas. Al provenir la música, en general, de entidades superiores, habría que
comunicarse con estas entidades también mediante esta música. 

Además, la distinta emotividad a la hora de expresarse, o una expresión rítmica constituye
otra forma de, si no música, sí elementos musicales, como son la interpretación o el
ritmo. Es decir, la música nació al prolongar y elevar los sonidos del lenguaje. 
 
“El arte de la música es el que más cercano se 
halla de las lágrimas y los recuerdos.”
 
En su sentido más amplio, la música nace con el ser humano, y ya estaba presente, es por
tanto una manifestación cultural universal.

La música se podía manifestar en la propia naturaleza o en las actividades cotidianas. Al
golpear dos piedras, o al cortar un árbol, se producía un sonido rítmico, el mantenimiento
de algo rítmico ayudaba a la realización de esa actividad. Pudo haber un primer grito o
palabra que servía como ánimo, apoyo, y para elaborar más eficazmente una determinada
actividad. Estos irían evolucionando a pequeñas frases, versos, hasta terminar ligándolos
en una canción.

\nomenclature[z-cif]{$CIF$}{Cauchy's Integral Formula}                                % first letter Z is for Acronyms 
\nomenclature[a-F]{$F$}{complex function}                                                   % first letter A is for Roman symbols
\nomenclature[g-p]{$\pi$}{ $\simeq 3.14\ldots$}                                             % first letter G is for Greek Symbols
\nomenclature[g-i]{$\iota$}{unit imaginary number $\sqrt{-1}$}                      % first letter G is for Greek Symbols
\nomenclature[g-g]{$\gamma$}{a simply closed curve on a complex plane}  % first letter G is for Greek Symbols
\nomenclature[x-i]{$\oint_\gamma$}{integration around a curve $\gamma$} % first letter X is for Other Symbols
\nomenclature[r-j]{$j$}{superscript index}                                                       % first letter R is for superscripts
\nomenclature[s-0]{$0$}{subscript index}                                                        % first letter S is for subscripts


%********************************** %Second Section  *************************************
\section{Evolución de la música	} %Section - 1.2
\subsection{Prehistoria}

Se desconoce cómo pudo ser la música en la Prehistoria, ya que no queda ningún registro
sonoro ni escrito de la misma. Pero sí que han ido apareciendo pequeños instrumentos, o la
evidencia de cierta tecnología gracias al arte mueble y al arte parietal que permite
pensar el que pudieran haber realizado instrumentos o que tuvieran el desarrollo
suficiente para crear música. A medida que vamos avanzando en el tiempo, vamos encontrando
elementos cada vez más complejos y que no establecen duda alguna de la presencia de
instrumentos en las sociedades prehistóricas y protohistóricas.

Los instrumentos musicales que se encuentran en la Prehistoria se pueden dividir en varios
grupos:
 
Autófonos: aquellos que producen sonidos por medio de la materia con la que están
construidos. Son instrumentos de percusión; por ejemplo, hueso contra piedra.

Membranófonos: serie de instrumentos más sencillos que los construidos por el hombre.

Tambores: hechos con una membrana tirante, sobre una nuez de coco, un recipiente
cualquiera o una verdadera y auténtica caja de resonancia.

Cordófonos: son aquellos de cuerda; por ejemplo, el arpa.

Aerófonos: el sonido se origina en ellos por vibraciones de una columna de aire. Uno de
los primeros instrumentos es la flauta, en un principio construida con un hueso con
agujeros.


\subsection{Paleolítico}

En el Paleolítico superior y más raramente en el Paleolítico Medio es donde encontramos
evidencias o indicios de la existencia tanto de primitivos instrumentos musicales como de
representaciones artísticas de los mismos. Desde finales del siglo XIX se viene publicando
la presencia de pitos o flautas encontradas en diversos yacimientos, pero solamente han
empezado a ser tomadas en serio y estudiadas en profundidad desde los años 60 del siglo
XX, en la actualidad, este sigue siendo un campo bastante polémico. Por un lado, hay
discusiones abiertas sobre si ciertos instrumentos estaban hechos para producir sonido, y
por otro lado si en realidad tienen un origen antrópico o son por el contrario el
resultado de depredadores y la erosión.


\subsection{Neolítico}

Escena de caza, en el Barranco de la Valltorta. Se aprecian los arcos y las cuerdas.

Tenemos muchos ejemplos en la pintura rupestre del periodo Neolítico de la existencia de
arcos. Aunque la mayoría se encuentran en contextos de caza, la realidad es que si
conocían la manera de construir un arco también sabían que una cuerda tensada a distintas
longitudes produce sonidos distintos.


\subsection{Antiguo Egipto y Mesopotamia}

La música en Egipto poseía avanzados conocimientos que eran reservados para los
sacerdotes. En el Imperio Nuevo utilizaban ya la escala de siete sonidos. Este pueblo
contó con un instrumentario rico y variado; algunos de los más representativos son el arpa
como instrumento de cuerdas y el oboe doble como instrumento de viento. En Mesopotamia los
músicos eran considerados personas de gran prestigio; acompañaban al monarca no solo en
los actos de culto, sino también en las suntuosas ceremonias de palacio y en las guerras.
El arpa es uno de los instrumentos más apreciados en Mesopotamia. La expresión musical de 
la Mesopotamia es considerada origen de la cultura musical occidental.


\subsection{Antigua Grecia}

En la Antigua Grecia, la música se vio influida por todas las civilizaciones que la
rodeaban, dada su importante posición estratégica. Culturas como la mesopotámica, etrusca,
egipcia o incluso las indoeuropeas fueron de importante influencia tanto en sus músicas
como en sus instrumentos musicales. Los griegos daban mucha importancia al valor educativo
y moral de la música. Por ello está muy relacionada con el poema épico. Aparecen los
bardos o aedos que, acompañados de una lira, vagan de pueblo en pueblo mendigando y
guardando memoria oral de la historia de Grecia y sus leyendas. Fue entonces cuando se
relaciona la música estrechamente con la filosofía. Los sabios de la época resaltan el
valor cultural de la música.


\subsection{Antigua Roma}

Habitualmente se utilizaba la música en las grandes fiestas. Eran muy valorados los
músicos virtuosos o famosos, añadiendo vertientes humorísticas y distendidas a sus
actuaciones. Estos músicos vivían de una manera bohemia, rodeados siempre de fiestas.

En los teatros romanos o anfiteatros se representaban comedias al estilo griego. La música
tenía un papel trascendental en estas obras teatrales.

A partir de la fundación de Roma sucede un hito musical, los ludiones. Estos eran unos
actores de origen etrusco que bailaban al ritmo de la tibia (una especie de instrumento
musical de viento). Los romanos intentan imitar estos artes y añaden el elemento de la
música vocal. A estos nuevos artistas se les denominó histriones que significa bailarines
en etrusco. Ninguna música de este estilo ha llegado hasta nosotros salvo un pequeño
fragmento de una comedia de Terencio.


\subsection{Edad Media}

Los orígenes de la música medieval se confunden con los últimos desarrollos de la música
del periodo tardío romano. La evolución de las formas musicales apegadas al culto se
resolvió a finales del siglo VI en el llamado canto gregoriano (se refiere en general a un
tipo de canto llano utilizado en la liturgia de la Iglesia católica, aunque en ocasiones
se usa en un sentido amplio o incluso como sinónimo de canto llano).


\subsection{Renacimiento}

Se caracteriza por una suave sonoridad que deriva de la aceptación de la tercera como
intervalo armónico consonante (uniéndose en esta categoría a quintas y octavas, ya
 admitidas en la Edad Media) y del progresivo aumento del número de voces, todas de igual
 importancia y regidas por las reglas del contrapunto: independencia de las voces,
 preparación y resolución de las disonancias, uso de terceras y sextas paralelas,
 exclusión de las quintas y octavas paralelas, etcétera.

El prototipo de obra musical renacentista es una pieza vocal de textura polifónica,
frecuentemente imitativa, escrita para entre tres y seis voces de carácter cantable; cada
línea melódica o voz podía ser interpretada indistintamente con voces reales o con
instrumentos. Si bien el rango de cada línea supera apenas la octava, la extensión general
del conjunto rebasa ampliamente las dos octavas, evitándose el cruce entre las voces (que
forzaba a que estas fueran heterogéneas y contrastantes en la polifonía medieval).

El sistema melódico utilizado siguió siendo el de los ocho modos gregorianos; las
características modales (opuestas a las tonales) de la música del Renacimiento comenzaron
a diluirse hacia el final del período con el uso creciente de intervalos de quinta como
movimiento entre fundamentales, característica definitoria de la tonalidad.


\subsection{Período de la práctica común o clásica}

En este período se desarrollaron nuevas formas y se operaron grandes avances técnicos
tanto en la composición como en el virtuosismo; así tenemos: cromatismo, expresividad,
bajo cifrado y bajo continuo, intensidad, ópera, oratorio, cantata, sonata, tocata, suite,
fuga y la sinfonía.


\subsection{Bajo contínuo}

También llamado bajo cifrado, permitía al compositor trazar tan solo el contorno de la
melodía  dejando las voces medias, es decir, el relleno armónico, a la invención del
continuista. La ejecución del continuo requiere dos instrumentistas: un instrumento
melódico grave (viola, gamba, violonchelo, contrabajo, fagot, etc.) que ejecuta las notas
del bajo y un instrumento armónico (laúd, clavecín, órgano) a cargo del continuista, quien
 desenvuelve improvisadamente las armonías, de acuerdo con las cifras del bajo cifrado, en
 la forma de acordes arpegios u otras figuraciones, todo ello de conforme al estilo y las
 necesidades expresivas del texto musical.


\subsection{Sistema tonal}

El sistema tonal fue una evolución desde los últimos maestros de la música medieval hasta
su máximo esplendor desde Bach a los últimos compositores tonales del posromanticismo. En
sus comienzos, se definió una armonía musical compuesta por siete asuntos distintos: Las
notas, los intervalos, los géneros, los sistemas de escala, los tonos, la modulación y la
composición de melodías.


\subsection{Clasicismo}

Es el estilo caracterizado por la transición de la música barroca hacia una música
equilibrada entre estructura y melodía. Ocupa la segunda mitad del siglo XVIII. Autores
que han marcado historia en este periodo fueron Carl Philipp Emanuel Bach, Franz Joseph
Haydn, Wolfgang Amadeus Mozart y Ludwig van Beethoven. 


\subsection{Romanticismo}

Aproximadamente desde 1830, lo romántico hace su aparición en toda Europa. Ha sido un
fenómeno que se fue instalando con el paso del tiempo. Se dice que en las últimas obras
de Mozart se puede apreciar algo de pre romanticismo.

El romanticismo es un movimiento que se inserta en todos los planos de la vida. Se 
caracteriza por el sentimiento, la pasión, imaginación, dominio del corazón sobre la 
cabeza y por el sentido de libertad.

La música romántica es apasionada, con frecuencia busca sonidos horrísonos para expresar
los más tremendos sentimientos


\subsection{Impresionismo}

La corriente dominante en Europa, influida por la mentalidad positivista, era ya el
realismo.

El impresionismo trata de representar no toda la realidad, sino lo que nos llama la
atención de la realidad, las sensaciones que nos produce. 

Los músicos llegaron un poco más tarde a este período pero también han encontrado una
manera de expresarse musicalmente. Buscaron sonidos líquidos, tímidos, decadentes,
calurosos. Buscaban sugerir cosas al público, darles un primer vistazo de lo que el músico
buscaba producir en su público.


\subsection{Contemporáneo}

Durante el siglo XX, sobre todo en la primera mitad, se ha ido perdiendo la relación entre 
los compositores y los inventores. Una de las posibles causas de la pérdida de esta
relación es que la música del siglo XX es poco comercial. La investigación por hallar
sonidos distintos es poco rentable.

Los nuevos sonidos, que los compositores necesitan para expresar lo que desean han
acabado obteniéndose no de instrumentos manuales, sino por medios electrónicos, hoy día
más al alcance de un músico que de los aparatos mecánicos. La nueva música tenía que
buscar su forma de expresión, y es el campo de la electrónica aquel en que se ha tratado
de encontrar la respuesta (no debemos confundir esto con el propio género de música
electrónica).


\nomenclature[z-DEM]{DEM}{Discrete Element Method}
\nomenclature[z-FEM]{FEM}{Finite Element Method}
\nomenclature[z-PFEM]{PFEM}{Particle Finite Element Method}
\nomenclature[z-FVM]{FVM}{Finite Volume Method}
\nomenclature[z-BEM]{BEM}{Boundary Element Method}
\nomenclature[z-MPM]{MPM}{Material Point Method}
\nomenclature[z-LBM]{LBM}{Lattice Boltzmann Method}
\nomenclature[z-MRT]{MRT}{Multi-Relaxation 
Time}
\nomenclature[z-RVE]{RVE}{Representative Elemental Volume}
\nomenclature[z-GPU]{GPU}{Graphics Processing Unit}
\nomenclature[z-SH]{SH}{Savage Hutter}
\nomenclature[z-CFD]{CFD}{Computational Fluid Dynamics}
\nomenclature[z-LES]{LES}{Large Eddy Simulation}
\nomenclature[z-FLOP]{FLOP}{Floating Point Operations}
\nomenclature[z-ALU]{ALU}{Arithmetic Logic Unit}
\nomenclature[z-FPU]{FPU}{Floating Point Unit}
\nomenclature[z-SM]{SM}{Streaming Multiprocessors}
\nomenclature[z-PCI]{PCI}{Peripheral Component Interconnect}
\nomenclature[z-CK]{CK}{Carman - Kozeny}
\nomenclature[z-CD]{CD}{Contact Dynamics}
\nomenclature[z-DNS]{DNS}{Direct Numerical Simulation}
\nomenclature[z-EFG]{EFG}{Element-Free Galerkin}
\nomenclature[z-PIC]{PIC}{Particle-in-cell}
\nomenclature[z-USF]{USF}{Update Stress First}
\nomenclature[z-USL]{USL}{Update Stress Last}
\nomenclature[s-crit]{crit}{Critical state}
\nomenclature[z-DKT]{DKT}{Draft Kiss Tumble}
\nomenclature[z-PPC]{PPC}{Particles per cell}