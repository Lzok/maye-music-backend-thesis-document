%!TEX root = ../thesis.tex
%*******************************************************************************
%****************************** Third Chapter **********************************
%*******************************************************************************
\chapter{Método}

% **************************** Define Graphics Path **************************
\ifpdf
    \graphicspath{{Chapter3/Figs/Raster/}{Chapter3/Figs/PDF/}{Chapter3/Figs/}}
\else
    \graphicspath{{Chapter3/Figs/Vector/}{Chapter3/Figs/}}
\fi

\section{Estudio de tecnologías}

\subsection{Proceso de aprendizaje}

Al comienzo del proceso, se dedicó el mes de Diciembre en el estudio de las tecnologías necesarias para el desarrollo de la aplicación.
Se comenzó con el micro-framework Lumen, basado en Laravel, herramienta principal con la cual se trabajaría. 
Ya que algunos integrantes ya tenían experiencia previa utilizando Laravel, pudo agilizarse el aprendizaje, necesitando solamente aprender sobre las mínimas diferencias que entre los dos presentan.

Se investigó sobre Vagrant para poder utilizar un entorno virtual de trabajo controlado y compartido por todos los integrantes, dejando de lado problemas de versión en la instalación de herramientas, o de compatibilidad con sistemas operativos diferentes.

Se adquierieron conocimientos sobre Automation Testing, para realizar tests de integración a los endpoints de la API, una vez se haya encontrado funcional.

Al iniciar Enero, se inició la investigación sobre los distintos servicios de APIs públicas que se utilizarían y procesarían, teniendo que aprender sobre el protocolo de autenticación OAuth2 para poder acceder a ellos.

Teniendo ya un buen nivel de control sobre los recursos necesarios, se comenzó a debatir sobre los posibles patrones o estructuras que serían utilizados para el acceso, proceso y respuesta de información obtenida de los servicios.

\subsection{Estado alcanzado}

\section{Aspectos clave}

\subsection{Estructura de conexión a servicios}

\subsection{Caché}

\subsection{Automation testing}

\subsection{Documentación de la API}

\section{Proceso de desarrollo}

\subsection{Sprint a...}

\subsection{...Sprint n}