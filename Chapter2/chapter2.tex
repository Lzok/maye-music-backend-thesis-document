%!TEX root = ../thesis.tex
%*******************************************************************************
%****************************** Second Chapter *********************************
%*******************************************************************************

\chapter{Marco teórico}

\ifpdf
    \graphicspath{{Chapter2/Figs/Raster/}{Chapter2/Figs/PDF/}{Chapter2/Figs/}}
\else
    \graphicspath{{Chapter2/Figs/Vector/}{Chapter2/Figs/}}
\fi


\section{Tecnologías}

\subsection{HTTP}

HTTP, Hypertext Transfer Protocol (Protocolo de Transferencia de Hipertexto) por sus
siglas, es el protocolo por el cual se comunican navegadores, servidores y aplicaciones
relacionadas con la web alrededor de todo el mundo.

Este protocolo se encuentra en la capa de aplicación del modelo OSI. Ha estado en uso por
la World-Wide Web (WWW) desde 1990. Su primera versión hace referencia a HTTP/0.9 y era un 
protocolo simple para la transferencia de datos a través de la internet.

HTTP/1.0 surge a través de la definición del RFC 1945, este mejoró sustancialmente la
comunicación permitiendo a los mensajes encontrarse en formato MIME (Multipurpose
Internet Mail Extensions). Los servidores web adjuntan un tipo MIME a todas sus peticiones
HTTP. Cuando un navegador web obtiene una respuesta desde un servidor, mira el tipo MIME
asociado a ella para ver si sabe cómo manejar el archivo. La mayoría de los navegadores
pueden manejar cientos de tipos de archivos populares: Imágenes, videos, sonidos,
documentos de texto, entre otros.

La última versión de este procolo es la 2.0, definida en el RFC 7540 en el año 2015. Un
gran avance en esta última versión contribuye a disminuir el tráfico innecesario en 
la red definiendo un mapeo optimizado de la semántica de HTTP a una conexión subyacente.




\subsection{API REST}

\subsection{AJAX}

\subsection{JWT}

\subsection{Memoria Cache}


\section{Herramientas y tecnologías utilizadas}

\subsection{PHP}

\subsection{Lumen Framework}

\subsection{SQL}

\subsection{Taiga}

\subsection{Vagrant}

\subsection{Git}